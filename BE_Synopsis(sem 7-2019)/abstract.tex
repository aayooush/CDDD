\begin{abstract}
\hspace{5mm} Abstract(max 300 words)Android OS devices such as phone, tablets are growing so faster in markets. Unfortunately, security is not concurrently guaranteed with functionality. Android has employed various mechanisms like application sand boxing, application signing,
permission model, etc. to enhance the security. But all these mechanisms have certain limitations.

\par The Bring Your Own Device paradigm allows mobile devices to connect with virtual network infrastructure which consist of federated servers in order to access to services and functionality. But basic security support by Android and other applications in market is insufficient for security requirements.
\par A security framework for android devices (specifically smart phones) where organizations or institutes decide security policies that employee must accept. Such rules and policies are aim to avoiding users to download any third party software inside the organization which invite outsider threats and even avoiding user to take sensitive information or data to outside which is nothing but insider threat.
\par Single -sign on server uses method of access control that enables user to log in only once and gain access to multiple resources without being prompted to log in again.
Single sign on server is centralized identity management  which authenticate the user by accepting user’s credentials.
Single sign on can be use with BYOD scenario for additional and one time authentication for android devices.


\end{abstract}
