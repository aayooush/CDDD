\graphicspath{ {images/} }
\chapter{Methodology}
%\subsection*{}
%\textbf{[1]}\\
\hspace{5mm} Research has been done to provide security framework for BYOD paradigm for employees and organization.BYOD security framework proposed and prototype using Mobile Device Management (MDM) but still industries not satisfied with security policies(due  to inside and outside threat).
Because of this reason ,BYOD security still in research.

\section{Methodology/Procedures} 
List to list all methodologies that you found during literature survey. Explain all algorithms that will be use in your project.

\hspace{5mm}In this section all algorithms or any other implementation procedure need to explain with proper flow.
\par describe every procedure or algorithm with necessary diagram.
\par also explain different technology or languages you used for implementation/Deployment.

\begin{table}[h!]
  \begin{center}
    \caption{Your first table.}
    \label{tab:table1}
    \begin{tabular}{l|c|r} % <-- Alignments: 1st column left, 2nd middle and 3rd right, with vertical lines in between
      \textbf{Value 1} & \textbf{Value 2} & \textbf{Value 3}\\
      $\alpha$ & $\beta$ & $\gamma$ \\
      \hline
      1 & 1110.1 & a\\
      \hline
      2 & 10.1 & b\\
      3 & 23.113231 & c\\
    \end{tabular}
  \end{center}
\end{table}

\begin{table}[ht]
\caption{Nonlinear Model Results} % title of Table
\centering % used for centering table
\begin{tabular}{c c c c} % centered columns (4 columns)
\hline\hline %inserts double horizontal lines
Case & Method\#1 & Method\#2 & Method\#3 \\ [0.5ex] % inserts table
%heading
\hline % inserts single horizontal line
1 & 50 & 837 & 970 \\ % inserting body of the table
2 & 47 & 877 & 230 \\
3 & 31 & 25 & 415 \\
4 & 35 & 144 & 2356 \\
5 & 45 & 300 & 556 \\ [1ex] % [1ex] adds vertical space
\hline %inserts single line
\end{tabular}
\label{table:nonlin} % is used to refer this table in the text
\end{table}

\subsection{Procedures} 
Some mathematical formulae
\\
\[ x^n + y^n = z^n \]
\\
Multiple integrals.
\\
$\iint_V \mu(u,v) \,du\,dv$$
\\
\\
Sums and products
\\
Sum $\sum_{n=1}^{\infty} 2^{-n} = 1$ inside text
\\
$$\sum_{n=1}^{\infty} 2^{-n} = 1$$
\\
\\
\[
    \binom{n}{k} = \frac{n!}{k!(n-k)!}
\]
\\
\\
When displaying fractions in-line, for example \(\frac{3x}{2}\) 
you can set a different display style: 
\( \displaystyle \frac{3x}{2} \).
 
This is also true the other way around
 
\[ f(x)=\frac{P(x)}{Q(x)} \ \ \textrm{and} 
\ \ f(x)=\textstyle\frac{P(x)}{Q(x)} \]