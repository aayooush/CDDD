\graphicspath{ {images/} }
\chapter{Introduction}
\thispagestyle{empty}
%The scope of the work is as follows:
\section{Introduction Description}
\par \hspace{5mm} Agriculture is the primary occupation in India. Crops are part and parcel of the agricultural industry. Nowadays, a tremendous loss in the quality and quantity of crop yield is observed, subject to various diseases in the plant. Crop Plant disease classification is a critical step, which can be useful in early detection of pest, insects, disease control, increase in productivity, among other examples. Farmers recognize disease manually with foregoing symptoms of plants, and with experts, whereas the actual diseases are hard to distinguish with naked eye, and it is time-consuming to predict whether the crop is healthy or not. Cotton is one of the major agricultural crops in India and it has a dominant impact on the overall Indian agriculture sector. Cotton plant leaf disease diagnosis is very difficult through observation to find the symptoms on plant leaves, incorporates it's a part of a high degree of complexity. Due to complexity even experienced agronomists and plant pathologists often fail to successfully diagnose specific diseases and are consequently led to mistaken conclusions and treatments.

\par Our study helps to predict crop diseases in cotton plants by processing the images of the crop. For this, Image Processing techniques are used for the very fast, accurate and appropriate classification of diseases. Symptoms of diseases in cotton predominantly come out on leaves of plants. The existence of an automated system for the detection and diagnosis of plant diseases would offer a support system to the agronomist who is performing such diagnosis through observation of the leaves of infected plants. 

\par The existing techniques for disease detection have utilized various image processing methods followed by various classification techniques. Crop Yield Forecasting has been an area of interest for producers, agricultural-related organizations. Timely and accurate crop yield forecasts are essential for crop production. The proposed system uses an artificial neural network to classify the health of a cotton leaf plant.   


%
\section{Organization of Report}
\begin{itemize}
\item \textbf{Ch.1 Introduction:} A web platform for crop disease detection purposes that helps the agricultural industry.
\item \textbf{Ch.2 Review of Literature:}Determining algorithms used for classification of the given crop images
\item \textbf{Ch.3 Proposed System:}Using Image Processing and Classification algorithms with help of neural networks to determine disease in crops
\item \textbf{Ch.4 Implementation Plan:} Using web upload to detect crop disease and obtain diagnosis of the crop condition
\item \textbf{Ch.5 Conclusion:}Detecting diseases in crops and determining solutions for effective prevention.

\end{itemize}
%
%\section{Layout of the Report} A brief chapter by chapter overview is presented here.\\
%Chapter 2: talks about the literature review used for the presented work. \\
%Chapter 3: gives a brief introduction about cloud computing and its various aspects.\\
%Chapter 4: discusses a brief overview of the various algorithms based on stochastic integer programming that were reviewed for this research. \\
%Chapter 5: the details of the cloud computing model and the various algorithms used for resource provisioning. Scenario reduction algorithms used for the modification of the cloud model is also explained in detail.  \\
%Chapter 6: describes the cloud computing model that was simulated using Java platform. The observations and comparisons of the resource provisioning cost obtained using this model are also presented.\\
%Chapter 7: the project road map is mentioned in this chapter, it includes various steps in literature review, simulation stage and the final stage of the work.\\
%Chapter 8: this chapter presents conclusion and future scope of the research work.\\