\graphicspath{ {images/} }
\chapter{Aims and Objectives}
\thispagestyle{empty}
%The scope of the work is as follows:
\par \hspace{5mm} The BYOD is a solution designed to specifically address the mobile enterprise needs. the system includes of key aspect called Mobile Device Management (MDM). It also supports single sign-on (SSO) and multi-tenancy. 

\par The server enables company/institutes or organizations to manage, secure and monitor Android devices (e.g., smart phones and tablet PCs). Users need to accept the policy agreement, which states all the actions that can be carried out on the device when enrolling with federated server or MDM server.Main advantage of this system is, it only controls the corporate data that is present on the devices, while the personal data is secure by pin number provided by employee or end user.i.e.personal data is kept untouched.

\par The administrator can create policies in on server and define the device management rules, applications which are blacklisted and application which are allowed and need to install on android device after perticuler policy get enforced. When employees register their devices with server, the applicable policy rules (e.g., enabling the phone lock, disabling the camera, and more) will be enforced on their devices. Every moment MDM server monitor the device and accordingly can select follow-up actions like sending the user a warning message, enforce the policy again, and more) based on their security requirements.  

\par The mobile management system consists of three key consoles: Console, Publisher and Store. Users uses the Publisher to manage enterprise apps throughout their application life cycle, which includes application states such as, published, unpublished, approved, rejected, deprecated, and retired. The Store acts as a marketplace and contains all the corporate mobile apps, which users can search, view, rate and install on-demand. The administrator uses the Console to manage users, administer and monitor policies.

\section{Motivation \& Aims}
\hspace{5mm} Describe your motivation behind developing this project.
\hspace{5mm} The Android platform is attracting organizations/institutes that plan to adopt Android for professional use. In general, this trend consists in using employees owned private devices, e.g., smart phones and tablet devices, inside organizations. This is due to the growing capabilities(mobile device configuration that match ) of smart phones,their diffusion and low costs.
In this context, the idea of “Bring Your Own Device(BYOD)” is developing a policy and enforcement mechanisms allowing persons to use their own personal device in a professional context.
\par Deploying the secure BYOD paradigm on Android requires major security issues that Security Framework should natively able to manage. At the same time employees must agree for such paradigm where they will face policies enforcement.
The main challenge in this system is secure framework where android devices securely used  within organization infrastructure without violating any policy so that organization service and data remain safe within organization only.  
 
%
\section{Objectives}
\hspace{5mm} The proposed system can be deployed as a mobile application and MDM server. Secure connection,single-sign on both can be used for deployment. Following objectives can be fulfilled with the help of the mobile application and MDM server:
\begin{itemize}
\item A user friendly android interface on which BYOD scenario can be implement.i.e. access
services through this interface.
\item Configuring and plug-ins for MDM servers which handle mobile devices,like managing
services and access control.
\item Preventing use of the organizations information on a device with applications not formerly
approved.i.e restrict downloading of third party software.
\item Encrypting the organizations data stored on a device to stop unauthorized applications
and users from accessing it.
\item Device locking is safeguarding of devices.By authorization of an administrator, enter-
prise server software can issue a command to immediately lock a managed mobile device
preventing access until the necessary credentials (such as passwords, biometrics or cryp-
tographic tokens) have been presented.
\item Remote wipes involves configuring a device so that after a certain number of consecutive
failed authentication attempts, the device will securely wipe itself.

\end{itemize}
